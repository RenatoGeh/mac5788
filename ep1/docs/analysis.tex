\documentclass[12pt]{article}

\usepackage[brazilian]{babel}
\usepackage[utf8]{inputenc}
\usepackage{graphicx}
\usepackage{mathtools}
\usepackage{amsthm}
\usepackage{amssymb}
\usepackage{thmtools,thm-restate}
\usepackage{amsfonts}
\usepackage{hyperref}
\usepackage[labelfont=bf,singlelinecheck=false]{caption}
\usepackage[backend=biber,url=true,doi=true,eprint=false,style=numeric]{biblatex}
\usepackage{enumitem}
\usepackage{indentfirst}
\usepackage{algorithm}
\usepackage[noend]{algpseudocode}
\usepackage{listings}
\usepackage[x11names,rgb,table]{xcolor}
\usepackage{tikz}
\usepackage{hyperref}
\usepackage{subcaption}
\usepackage{booktabs}
\usepackage{linegoal}
\usepackage{geometry}
\usetikzlibrary{snakes,arrows,shapes}

\addbibresource{references.bib}
\graphicspath{{imgs/}}

\makeatletter
\def\subsection{\@startsection{subsection}{3}%
  \z@{.5\linespacing\@plus.7\linespacing}{.1\linespacing}%
  {\normalfont}}
\makeatother

\makeatletter
\patchcmd{\@setauthors}{\MakeUppercase}{}{}{}
\makeatother

\DeclareMathOperator*{\argmin}{arg\,min}
\DeclareMathOperator*{\argmax}{arg\,max}
\DeclareMathOperator*{\Val}{\text{Val}}
\DeclareMathOperator*{\Ch}{\text{Ch}}
\DeclareMathOperator*{\Pa}{\text{Pa}}
\DeclareMathOperator*{\Sc}{\text{Sc}}
\newcommand{\ov}{\overline}
\newcommand{\tsup}{\textsuperscript}

\newcommand\defeq{\mathrel{\overset{\makebox[0pt]{\mbox{\normalfont\tiny\sffamily def}}}{=}}}

\newcommand{\algorithmautorefname}{Algorithm}
\algrenewcommand\algorithmicrequire{\textbf{Entrada}}
\algrenewcommand\algorithmicensure{\textbf{Saída}}
\algrenewcommand\algorithmicif{\textbf{se}}
\algrenewcommand\algorithmicthen{\textbf{então}}
\algrenewcommand\algorithmicelse{\textbf{senão}}
\algrenewcommand\algorithmicfor{\textbf{para todo}}
\algrenewcommand\algorithmicdo{\textbf{faça}}
\algnewcommand{\LineComment}[1]{\State\,\(\triangleright\) #1}

\theoremstyle{plain}

\newcounter{dummy-def}\numberwithin{dummy-def}{section}
\newtheorem{definition}[dummy-def]{Definition}
\newcounter{dummy-thm}\numberwithin{dummy-thm}{section}
\newtheorem{theorem}[dummy-thm]{Theorem}
\newcounter{dummy-prop}\numberwithin{dummy-prop}{section}
\newtheorem{proposition}[dummy-prop]{Proposition}
\newcounter{dummy-corollary}\numberwithin{dummy-corollary}{section}
\newtheorem{corollary}[dummy-corollary]{Corollary}
\newcounter{dummy-lemma}\numberwithin{dummy-lemma}{section}
\newtheorem{lemma}[dummy-lemma]{Lemma}
\newcounter{dummy-ex}\numberwithin{dummy-ex}{section}
\newtheorem{exercise}[dummy-ex]{Exercise}
\newcounter{dummy-eg}\numberwithin{dummy-eg}{section}
\newtheorem{example}[dummy-eg]{Example}

\numberwithin{equation}{section}

\newcommand{\set}[1]{\mathbf{#1}}
\newcommand{\pr}{\text{P}}
\newcommand{\eps}{\varepsilon}
\newcommand{\ddspn}[2]{\frac{\partial#1}{\partial#2}}
\newcommand{\iddspn}[2]{\partial#1/\partial#2}
\newcommand{\indep}{\perp}
\renewcommand{\implies}{\Rightarrow}

\newcommand{\bigo}{\mathcal{O}}

\setlength{\parskip}{1em}

\lstset{frameround=fttt,
	numbers=left,
	breaklines=true,
	keywordstyle=\bfseries,
	basicstyle=\ttfamily,
}

\newcommand{\code}[1]{\lstinline[mathescape=true]{#1}}
\newcommand{\mcode}[1]{\lstinline[mathescape]!#1!}

\title{%
  Análise da Simetria no Jogo da Velha\\~\\
  {\normalfont MAC5788 Planejamento em Inteligência Artificial}
}
\author{Renato Lui Geh, NUSP: 8536030}
\date{}

\begin{document}

\maketitle

\section{Simetria}

Diz-se que duas configurações do jogo da velha são simétricas quando os dois tabuleiros são
idênticos a menos de rotação e reflexão. Podemos considerar o tabuleiro como uma matriz $3\times
3$, onde a marcação do primeiro jogador é representado por $1$, o do segundo por $2$ e uma posição
vazia por $0$.\\

\begin{table}[h]
  \centering
  \begin{tabular}{c|c|c}
    1 & 0 & 0\\
    \hline
    0 & 2 & 0\\
    \hline
    0 & 0 & 1\\
  \end{tabular}
  \qquad\qquad
  \begin{tabular}{c|c|c}
    X &  & \\
    \hline
     & O & \\
    \hline
     &  & X\\
  \end{tabular}
  \qquad\qquad
  \begin{tabular}{c|c|c}
    0 & 1 & 2\\
    \hline
    3 & 4 & 5\\
    \hline
    6 & 7 & 8\\
  \end{tabular}
  \caption{Configuração do tabuleiro em matriz (esquerda), no seu formato tradicional (meio), e os
  índices de cada posição no tabuleiro (direita).}
\end{table}

Desta forma, toda configuração do jogo é única e pode ser representada por um \emph{hash} único

\begin{equation}
  H(T)=\sum_{i=0}^8 T_i\cdot 3^i,\label{eq:hash}
\end{equation}

onde $T_i$ é a $i$-ésima posição do tabuleiro. É fácil de ver que $H(T)$ é apenas uma representação
da matriz em ternário.

Nosso objetivo é achar todas as configurações de tabuleiro simétricas e atribuirmos o mesmo valor
de hash a elas. Para isso, precisamos achar todos os possíveis jogos válidos, e para cada um
desses, acharmos seus equivalentes simétricos. Esta tarefa é fácil: basta acharmos todas as
possíveis combinações de reflexão e rotação. Vamos usar a notação $H(T)$ para representar o hash
único do tabuleiro $T$, dado pela~\autoref{eq:hash}. Além disso, considere um dicionário global $A$
que guarda todos os hashes vistos até agora.

\begin{algorithm}[H]
  \caption*{$\hat{H}(T)$: acha todas configurações simétricas de um tabuleiro $T$\label{alg:sym}}
  \begin{algorithmic}[1]
    \Require configuração $T$ do tabuleiro
    \Ensure hash único de todos os tabuleiros simétricos a $T$
    \State $m\gets\infty$
    \State $h\gets H(T)$
    \If{$h\in A$}
      \State \textbf{retorna} $A[h]$
    \EndIf%
    \State Seja $U$ um vetor de hashes ainda não vistos
    \State Seja $M$ a matriz $3\times 3$ dada por $T$
    \For{$i\gets 0..3$}
      \State Rotaciona $M$ em 90 graus
      \State $h\gets H(M)$
      \If{$h\not\in A$}
        \State Adiciona $h$ em $U$
      \EndIf%
      \State $m\gets \min\{m,h\}$
      \State Reflete $M$
      \State $h\gets H(M)$
      \If{$h\not\in A$}
        \State Adiciona $h$ em $U$
      \EndIf%
      \State $m\gets \min\{m,h\}$
      \State Reflete $M$
    \EndFor%
    \For{todo $u\in U$}
      \State $A[u]\gets m$
    \EndFor%
    \State \textbf{retorna} $m$
  \end{algorithmic}
\end{algorithm}

Por meio de $\hat{H}(T)$, calculamos todas as configurações simétricas do tabuleiro $T$, guardamos
o hash de menor valor, garantindo que seja único a menos de rotações e reflexão, e em seguida
usamos programação dinâmica para evitar recomputar os hashes novamente. Deste modo, quando o agente
avaliar um estado correspondente ao tabuleiro $T$, ele irá considerar todas as simetrias ao mesmo
tempo.

\section{Código}

O código original pode ser encontrado em~\cite{ttt}, enquanto que o código alterado para simetria
em~\cite{mod-ttt}. A função em Python equivalente a $\hat{H}$ é o método \code{State.hash}. Para a
função $H$ de hash único, usa-se a função \code{State.unique_hash}. O dicionário $A$ com todos os
hashes apontando para seus simétricos é a variável global \code{all_hashes}.

Além da função de hash e funções auxiliares, nada mais foi alterado.

\section{Experimentos}

O jogo da velha possue uma estratégia vencedora para o primeiro jogador, vencendo ou empatando se
jogar otimamente.

\begin{table}[h]
  \centering
  \begin{tabular}{c|c|c}
    X &  & \\
    \hline
     &  & \\
    \hline
     &  & \\
  \end{tabular}
  \quad$\rightarrow$\quad
  \begin{tabular}{c|c|c}
    X &  & \\
    \hline
     &  & \\
    \hline
     &  & X\\
  \end{tabular}
  \quad$\rightarrow$\quad
  \begin{tabular}{c|c|c}
    X &  & X\\
    \hline
     &  & \\
    \hline
     &  & X\\
  \end{tabular}
  \caption{Estratégia vencedora do primeiro jogador. Todas estratégias vencedoras simétricas podem
  ser reduzidas à estratégia acima por meio de rotações ou reflexões.}
\end{table}

Na versão sem exploração da simetria do código original, o segundo jogador ou empatava ou perdia
quando confrontado com a estratégia vencedora do primeiro jogador. Em contrapartida, o agente
treinado com exploração da simetria sempre empatava contra um primeiro jogador ótimo.

Em questão de tempo de convergência, enquanto que a versão sem simetria convergiu na época 65000, a
com simetria conseguiu a mesma convergência na época 6000, o que indica que o agente exaustou todas
as jogadas mais rapidamente, o que era esperado.

\printbibliography[]

\end{document}
